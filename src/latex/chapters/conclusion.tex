\doublespacing

We developed a method for quickly incorporating realistic effects into Monte Carlo (MC) calculations of fractional quantum Hall (FQH) energy gaps. We applied our method to the effect Landau level mixing (LLM) at electron filling factor $\nu=1/3$ in the lowest Landau level (LLL) of graphene. We began by fitting a set of equations (Eqs.~\ref{eq:first_pp_corr} -~\ref{eq:last_pp_corr}) to data of two-body LLM Haldane pseudopotential (PP) corrections calculated by Arciniaga \textit{et al.} in Ref. \cite{arciniaga}. We perturbatively added (Eq.~\ref{potLlm}) these corrections to two-body PPs in the spherical geometry generated by applying the Wooten formula (Eq.~\ref{wootPpN0}) to the bare Coulomb potential. We chose a modified Park potential (Eq.~\ref{modPark}) as our effective real space potential, which contains parameters that allow it to be mapped via the Wooten formula to the aforementioned LLM-incorporated PPs. We generated a data set of these fitting parameters for magnetic monopole strengths $Q$ and LLM parameters $\kappa$ for the following systems of interest (see Sec.~\ref{ssec:apprEffPotPar}): $Q\in$ \{4.5, 6.0, 7.5, 9.0, 10.5, 12.0, 13.5, 15.0, 18.0, 22.5, 28.5, 36.0, 43.5, 58.5, 73.5\} and $\kappa\in$ \{0.0, 0.1, 0.2, 0.5, 0.8, 0.9, 2.2\}. We fit to this data equations (Eqs.~\ref{eqn:paramEq} -~\ref{eqn:lastPar}) for the parameters in terms of $\kappa$ and/or $Q$ so that the effective real space potential can be generated automatically by the MC code for each system. We used these approximated parameters to calculate the composite fermion (CF)-exciton dispersion in the MC code, which then prompted us to change the probability weight function used for acceptance sampling from the ground state wavefunction to the wavefunction of the lowest energy state at the maximum total angular momentum ($L_{max}=N$). We then benchmarked this dispersion against the result from exact diagonalization and found that the MC transport gaps increased with increasing $\kappa$, in opposition to the trend of the benchmark. We tried a number of different techniques to alleviate this problem but ultimately concluded that for a reasonable number of MC iterations, the states most affected by LLM are too improbable to be sampled at the relative frequency required by the Metropolis-Hastings algorithm. Future studies need to be done to develop an algorithm that will sample the less likely, more dense electron configurations so that realistic FQH energy gaps in graphene can be calculated in the thermodynamic limit as a function of $0.5<\kappa\leq2.2$ at $\nu\in$ \{1/3,2/5,3/7,4/9,5/11,...\}; eventually including three-body particle-hole symmetry breaking terms in higher LLs and novel effects in other materials.

\singlespacing